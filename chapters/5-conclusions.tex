\chapter{Conclusions}
\label{ch:conclusions}
In this project, the existing Bird Daemon's integration package has been reviewed and refactored making it more robust and compliant with the latest OpenWrt/LEDE-supported Bird API. New graphical configuration pages have been added in order to add missing critical functionality highly used in most of Bird configurations. 
These new pages integrate some functionalities that, in the previous version, were forcing administrators to do manual changes in command line, thus being a big improvement to make this tool more user-friendly. Moreover, because one of the biggest challenges during this project has been the lack of documentation in some of the areas that were being improved, this dissertation has aimed to include reference information to facilitate its understanding and the Package's documentation has been updated and extended.
Finally, the theoretical analysis performed has shown a promising integration opportunity to enhance the Package's capabilities adding monitoring data.

This project's objectives have been successfully achieved on schedule even with the challenges occasioned due to the lack of official documentation. Even more, as an extra task for this project to prove its correctness, a network section has been deployed in the real environment (using virtual nodes) following the same topology and challenges as the targeted one. The network's configuration has been a real challenge as the environment and the management system were in Catalonia and the connection was done through a VPN in the UK. Moreover, because of the number of different technologies being used to get connection between both endpoints, it has required weeks of work to configure it as expected and to be able to get reliable data from it. Therefore, because this task was started after achieving all the requirements, it has not been possible to monitor it and to analyse the data but to confirm its correct behaviour.

Regarding this project's schedule and methodology followed, I have worked in a kanban-like manner, which has helped me focusing in one requirement at a time and also has been positive for V\'{i}ctor who, as a Stakeholder, received regular updates, new features demos and progress reports to help him manage the project and find possible risks or blockers. One of the recurring risks we did flag was the above mentioned network deployment. Because of its late deployment and  big number of unknowns, it did cause a big unplanned overhead on the project and its delivery.

\section{Future work}
This project has opened a number of future goals that have been recorded either in this dissertation or in the Package's documentation repository:

\begin{itemize}
    \item Finish Package repository's wrapping up in order to deliver this improvements to OpenWrt/LEDE's official stream.
    \item Continue improving Bird's integration by enhancing the Package (e.g. add OSPF protocol's graphical integration).
    \item Finish requirements' definition on uBus integration and UI capabilities foreseen to include from Bird's live data gathering and which implications it has on the system (i.e. performance or storage issues if we keep too much historical data).
    \item Analyse latest Bird Daemon v2.0.0 (currently in alpha state) and plan for any API disruptive change, new features and capabilities to take advantage of it while improving the current version.
\end{itemize}











 

