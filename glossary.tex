\newglossaryentry{xoln}{
name={XOLN},
description={: el Comuns de la Xarxa Oberta, Lliure i Neutral document. Guifi.net's principles to guarantee a Free, Open and Neutral Network. This principles guide both users and service providers how to interact in a fair manner. La Fundaci\'{o} Guifi.net is network's institution to arbitrate any issue arisen due to this principles non-compliance}
}

\newglossaryentry{igp}{
name={Interior Gateway Protocol},
description={: routing protocols being used in internal Autonomous Systems (single-administrated network behaving as an entity) to manage their connectivity and paths}
}

\newglossaryentry{egp}{
name={Exterior Gateway Protocol},
description={: routing protocols managing connectivity and paths on networks compound by Autonomous System entities}
}

\newglossaryentry{gsoc}{
name={Google Summer of Code},
description={: annual open source software development program (6 months) driven and sponsored by Google to support students looking for OSS projects and getting rewarded for developing them}
}

\newglossaryentry{taiga}{
name={Taiga.io},
description={: online project management tool working either with Kanban or SCRUM Agile methodologies. This tool is widely used in OSS projects due to its power, simplicity and plugins (open API) and has also enterprise options (\href{https://taiga.io/}{https://taiga.io/})}
}

\newglossaryentry{markd}{
name={MarkDown},
description={: programmatic documentation syntax to convert plain text documentation into HTML. There are several different implementations with slight differences. \href{https://daringfireball.net/projects/markdown/}{Original specification - 2004}}
}

\newglossaryentry{busybox}{
name={BusyBox},
description={: light and optimised UNIX tools including most of the widely used terminal commands (e.g. \texttt{ash}, \texttt{cat} or \texttt{rm}). Main \href{https://busybox.net/downloads/BusyBox.html}{documentation}}
}

\newglossaryentry{jsonc}{
name={JSON-C},
description={: a JSON implementation for parsing C language data structures. This implementation states that it is RFC7159 compliant.}
}

\newacronym{kvm}{KVM}{Kernel-based Virtual Machine}

\newacronym{cli}{CLI}{Command Line Interface}



